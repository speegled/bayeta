2:55:18.4	Johnny Hayes	 United States	July 24, 1908	London Olympics, England	IAAF[53]	Time was officially recorded as 2:55:18 2/5.[54] Italian Dorando Pietri finished in 2:54:46.4, but was disqualified for receiving assistance from race officials near the finish.[55] Note.[56]
2:52:45.4	Robert Fowler	 United States	January 1, 1909	Yonkers,[nb 5] United States	IAAF[53]	Note.[56]
2:46:52.8	James Clark	 United States	February 12, 1909	New York City, United States	IAAF[53]	Note.[56]
2:46:04.6	Albert Raines	 United States	May 8, 1909	New York City, United States	IAAF[53]	Note.[56]
2:42:31.0	Henry Barrett	 United Kingdom	May 8, 1909[nb 6]	Polytechnic Marathon, London, England	IAAF[53]	Note.[56]
2:40:34.2	Thure Johansson	 Sweden	August 31, 1909	Stockholm, Sweden	IAAF[53]	Note.[56]
2:38:16.2	Harry Green	 United Kingdom	May 12, 1913	Polytechnic Marathon	IAAF[53]	Note.[61]
2:36:06.6	Alexis Ahlgren	 Sweden	May 31, 1913	Polytechnic Marathon	IAAF[53]	Report in The Times claiming world record.[62] Note.[61]
2:38:00.8	Umberto Blasi	 Italy	November 29, 1914	Legnano, Italy	ARRS[10]	
2:32:35.8	Hannes Kolehmainen	 Finland	August 22, 1920	Antwerp Olmpics, Belgium	IAAF,[53] ARRS[10]	The course distance was officially reported to be 42,750 meters/26.56 miles,[63] however, the Association of Road Racing Statisticians estimated the course to be 40 km.[31]
2:29:01.8	Albert Michelsen	 United States	October 12, 1925	Port Chester Marathon, United States	IAAF[53]	Note.[64][65]
2:30:57.6	Harry Payne	 United Kingdom	July 5, 1929	AAA Championships, London, England	ARRS[10]	
2:26:14	Sohn Kee-chung	Japanese Korea	March 21, 1935	Tokyo, Japan	ARRS[10]	Also romanized as Kitei Son.
2:27:49.0	Fusashige Suzuki	 Japan	March 31, 1935	Tokyo, Japan	IAAF[53]	According to the Association of Road Racing Statisticians, Suzuki's 2:27:49 performance occurred in Tokyo on March 21, 1935, during a race in which he finished second to Sohn Kee-chung (sometimes referred to as Kee-Jung Sohn or Son Kitei) who ran a 2:26:14.[66]
2:26:44.0	Yasuo Ikenaka	 Japan	April 3, 1935	Tokyo, Japan	IAAF[53]	Note.[67]
2:26:42	Sohn Kee-chung	Japanese Korea	November 3, 1935	Meiji Shrine Games, Tokyo, Japan	IAAF[53]	Also romanized as Kitei Son. Note.[67]
2:25:39	Suh Yun-bok	 Korea	April 19, 1947	Boston Marathon	IAAF[53]	Disputed (short course).[68] Disputed (point-to-point).[69] Note.[70]
2:20:42.2	Jim Peters	 United Kingdom	June 14, 1952	Polytechnic Marathon	IAAF,[53] ARRS[10]	MarathonGuide.com states the course was slightly long.[71] Report in The Times claiming world record.[72]
2:18:40.4	Jim Peters	 United Kingdom	June 13, 1953	Polytechnic Marathon	IAAF,[53] ARRS[10]	Report in The Times claiming world record.[72]
2:18:34.8	Jim Peters	 United Kingdom	October 4, 1953	Turku Marathon	IAAF,[53] ARRS[10]	
2:17:39.4	Jim Peters	 United Kingdom	June 26, 1954	Polytechnic Marathon	IAAF[53]	Point-to-point course.[citation needed] Report in The Times claiming world record.[73]
2:18:04.8	Paavo Kotila	 Finland	August 12, 1956	Finnish Athletics Championships, Pieksämäki, Finland	ARRS[10]	
2:15:17.0	Sergei Popov	 Soviet Union	August 24, 1958	European Athletics Championships, Stockholm, Sweden	IAAF,[53] ARRS[10]	The ARRS notes Popov's extended time as 2:15:17.6[10]
2:15:16.2	Abebe Bikila	 Ethiopia	September 10, 1960	Rome Olympics, Italy	IAAF,[53] ARRS[10]	World record fastest marathon run in bare feet.[74]
2:15:15.8	Toru Terasawa	 Japan	February 17, 1963	Beppu-Ōita Marathon	IAAF,[53] ARRS[10]	
2:14:28	Leonard Edelen	 United States	June 15, 1963	Polytechnic Marathon	IAAF[53]	Point-to-point course.[citation needed] Report in The Times claiming world record and stating that the course may have been long.[75]
2:14:43	Brian Kilby	 United Kingdom	July 6, 1963	Port Talbot, Wales	ARRS[10]	
2:13:55	Basil Heatley	 United Kingdom	June 13, 1964	Polytechnic Marathon	IAAF[53]	Point-to-point course.[citation needed] Report in The Times claiming world record.[76]
2:12:11.2	Abebe Bikila	 Ethiopia	October 21, 1964	Tokyo Olympics, Japan	IAAF,[53] ARRS[10]	
2:12:00	Morio Shigematsu	 Japan	June 12, 1965	Polytechnic Marathon	IAAF[53]	Point-to-point course.[citation needed] Report in The Times claiming world record.[77]
2:09:36.4	Derek Clayton	 Australia	December 3, 1967	Fukuoka Marathon	IAAF,[53] ARRS[10]	
2:08:33.6	Derek Clayton	 Australia	May 30, 1969	Antwerp, Belgium	IAAF[53]	Disputed (short course).[78]
2:09:28.8	Ron Hill	 United Kingdom	July 23, 1970	Edinburgh Commonwealth Games, Scotland	ARRS[10]	
2:09:12	Ian Thompson	 United Kingdom	January 31, 1974	Christchurch Commonwealth Games, New Zealand	ARRS[10]	
2:09:05.6	Shigeru So	 Japan	February 5, 1978	Beppu-Ōita Marathon	ARRS[10]	
2:09:01	Gerard Nijboer	 Netherlands	April 26, 1980	Amsterdam Marathon	ARRS[10]	
2:08:18	Robert De Castella	 Australia	December 6, 1981	Fukuoka Marathon	IAAF,[53] ARRS[10]	
2:08:05	Steve Jones	 United Kingdom	October 21, 1984	Chicago Marathon	IAAF,[53] ARRS[10]	
2:07:12	Carlos Lopes	 Portugal	April 20, 1985	Rotterdam Marathon	IAAF,[53] ARRS[10]	
2:06:50	Belayneh Dinsamo	 Ethiopia	April 17, 1988	Rotterdam Marathon	IAAF,[53] ARRS[10]	
2:06:05	Ronaldo da Costa	 Brazil	September 20, 1998	Berlin Marathon	IAAF,[53] ARRS[10]	First time the 40K mark was passed under two hours (1:59:55).[79]
2:05:42	Khalid Khannouchi	 Morocco	October 24, 1999	Chicago Marathon	IAAF,[53] ARRS[10]	
2:05:38	Khalid Khannouchi	 United States	April 14, 2002	London Marathon	IAAF,[53] ARRS[10]	First "World's Best" recognized by the International Association of Athletics Federations.[80] The ARRS notes Khannouchi's extended time as 2:05:37.8[10]
2:04:55	Paul Tergat	 Kenya	September 28, 2003	Berlin Marathon	IAAF,[53] ARRS[10]	First world record for the men's marathon ratified by the International Association of Athletics Federations.[81]
2:04:26	Haile Gebrselassie	 Ethiopia	September 30, 2007	Berlin Marathon	IAAF,[53] ARRS[10]	
2:03:59	Haile Gebrselassie	 Ethiopia	September 28, 2008	Berlin Marathon	IAAF,[53] ARRS[10]	The ARRS notes Gebrselassie's extended time as 2:03:58.2.[10] Video on YouTube
2:03:38	Patrick Makau	 Kenya	September 25, 2011	Berlin Marathon	IAAF,[82][83] ARRS[84]	
2:03:23	Wilson Kipsang	 Kenya	September 29, 2013	Berlin Marathon	IAAF[85][86] ARRS[84]	The ARRS notes Kipsang's extended time as 2:03:22.2[84]
2:02:57	Dennis Kimetto	 Kenya	September 28, 2014	Berlin Marathon	IAAF[87][88] ARRS[84]	The ARRS notes Kimetto's extended time as 2:02:56.4[84]
2:01:39	Eliud Kipchoge	 Kenya	September 16, 2018	Berlin Marathon	IAAF[89]	
2:01:09	Eliud Kipchoge	 Kenya	September 25, 2022	Berlin Marathon	World Athletics[90]	
2:00:35	Kelvin Kiptum	 Kenya	October 8, 2023	Chicago Marathon	World Athletics[91]	First man to break 2:01:00 in a record-eligible marathon.
3:40:22	Violet Piercy	 United Kingdom	October 3, 1926	London [nb 7]	IAAF[53]	The ARRS indicates that Piercy's 3:40:22 was set on August 2, 1926, during a time trial on a course that was only 35.4 km.[10]
3:37:07	Merry Lepper	 United States	December 16, 1963[nb 8]	Culver City, United States	IAAF[53]	Disputed (short course).[95]
3:27:45	Dale Greig	 United Kingdom	May 23, 1964	Ryde	IAAF,[53] ARRS[10]	
3:19:33	Mildred Sampson	 New Zealand	July 21, 1964[nb 9]	Auckland, New Zealand	IAAF[53]	Disputed by ARRS as a time trial.[nb 9][98]
3:14:23	Maureen Wilton	 Canada	May 6, 1967	Toronto, Canada	IAAF,[53] ARRS[10]	The ARRS notes Wilton's extended time as 3:14:22.8[10]
3:07:27.2	Anni Pede-Erdkamp	 West Germany	September 16, 1967	Waldniel, West Germany	IAAF,[53] ARRS[10]	The ARRS notes Pede-Erdkamp's extended time as 3:07:26.2[10]
3:02:53	Caroline Walker	 United States	February 28, 1970	Seaside, OR	IAAF,[53] ARRS[10]	
3:01:42	Elizabeth Bonner	 United States	May 9, 1971	Philadelphia, United States	IAAF,[53] ARRS[10]	
2:55:22	Elizabeth Bonner	 United States	September 19, 1971	New York City Marathon	IAAF,[53] ARRS[10]	
2:49:40	Cheryl Bridges	 United States	December 5, 1971	Culver City, United States	IAAF,[53] ARRS[10]	
2:46:36	Michiko Gorman	 United States	December 2, 1973	Culver City, United States	IAAF,[53] ARRS[10]	The ARRS notes Gorman's extended time as 2:46:37[10]
2:46:24	Chantal Langlacé	 France	October 27, 1974	Neuf-Brisach, France	IAAF,[53] ARRS[10]	
2:43:54.5	Jacqueline Hansen	 United States	December 1, 1974	Culver City, United States	IAAF,[53] ARRS[10]	The ARRS notes Hansen's extended time as 2:43:54.6[10]
2:42:24	Liane Winter	 West Germany	April 21, 1975	Boston Marathon	IAAF[53]	Disputed (point-to-point).[69]
2:40:15.8	Christa Vahlensieck	 West Germany	May 3, 1975	Dülmen	IAAF,[53] ARRS[10]	
2:38:19	Jacqueline Hansen	 United States	October 12, 1975	Nike OTC Marathon, Eugene, United States	IAAF,[53] ARRS[10]	
2:35:15.4	Chantal Langlacé	 France	May 1, 1977	Oiartzun, Spain	IAAF[53]	
2:34:47.5	Christa Vahlensieck	 West Germany	September 10, 1977	Berlin Marathon	IAAF,[53] ARRS[10]	
2:32:29.8	Grete Waitz	 Norway	October 22, 1978	New York City Marathon	IAAF[53]	Disputed (short course).[50][101]
2:27:32.6	Grete Waitz	 Norway	October 21, 1979	New York City Marathon	IAAF[53]	Disputed (short course).[50][102]
2:31:23	Joan Benoit	 United States	February 3, 1980	Auckland, New Zealand	ARRS[10]	
2:30:57.1	Patti Catalano	 United States	September 6, 1980	Montreal, Canada	ARRS[10]	
2:25:41.3	Grete Waitz	 Norway	October 26, 1980	New York City Marathon	IAAF[53]	Disputed (short course).[50][103]
2:30:27	Joyce Smith	 United Kingdom	November 16, 1980	Tokyo, Japan	ARRS[10]	
2:29:57	Joyce Smith	 United Kingdom	March 29, 1981	London Marathon	ARRS[10]	
2:25:28	Allison Roe	 New Zealand	October 25, 1981	New York City Marathon	IAAF[53]	Disputed (short course).[50][104]
2:29:01.6	Charlotte Teske	 West Germany	January 16, 1982	Miami, United States	ARRS[10]	
2:26:12	Joan Benoit	 United States	September 12, 1982	Nike OTC Marathon, Eugene, United States	ARRS[10]	
2:25:28.7	Grete Waitz	 Norway	April 17, 1983	London Marathon	IAAF,[53] ARRS[10]	
2:22:43	Joan Benoit	 United States	April 18, 1983	Boston Marathon	IAAF[53]	Disputed (point-to-point).[69]
2:24:26	Ingrid Kristiansen	 Norway	May 13, 1984	London Marathon	ARRS[10]	
2:21:06	Ingrid Kristiansen	 Norway	April 21, 1985	London Marathon	IAAF,[53] ARRS[10]	
2:20:47	Tegla Loroupe	 Kenya	April 19, 1998	Rotterdam Marathon	IAAF,[53] ARRS[10]	
2:20:43	Tegla Loroupe	 Kenya	September 26, 1999	Berlin Marathon	IAAF,[53] ARRS[10]	
2:19:46	Naoko Takahashi	 Japan	September 30, 2001	Berlin Marathon	IAAF,[53] ARRS[10]	
2:18:47	Catherine Ndereba	 Kenya	October 7, 2001	Chicago Marathon	IAAF,[53] ARRS[10]	
2:17:18	Paula Radcliffe	 United Kingdom	October 13, 2002	Chicago Marathon	IAAF,[53] ARRS[10]	First "World's Best" recognized by the International Association of Athletics Federations.[80] The ARRS notes Radcliffe's extended time as 2:17:17.7[10]
2:15:25 Mx	Paula Radcliffe	 United Kingdom	April 13, 2003	London Marathon	IAAF,[53] ARRS[10]	First world record for the women's marathon ratified by the International Association of Athletics Federations.[105] The ARRS notes Radcliffe's extended time as 2:15:24.6[10]
2:17:42 Wo	Paula Radcliffe	 Great Britain	April 17, 2005	London Marathon	IAAF[106]	
2:17:01 Wo	Mary Jepkosgei Keitany	 Kenya	April 23, 2017	London Marathon	IAAF[107]	
2:14:04 Mx	Brigid Kosgei	 Kenya	October 13, 2019	Chicago Marathon	IAAF[108]	
2:11:53 Mx	Tigst Assefa	 Ethiopia	September 24, 2023	Berlin Marathon	World Athletics[109]	First woman to break the 2:12:00 barrier in the marathon.[110]